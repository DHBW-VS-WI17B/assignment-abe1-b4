Im Laufe der Zeit schreitet die Technologie im Bereich der Prozessoren von Servern und Computern immer weiter voran.
Die Prozessoren werden zwar nicht schneller, haben jedoch immer mehr Kerne.\ifootcite[1]{gruntz2010aktorenmodel}
Um die verschiedenen Kerne der Hardware ausnutzen zu können, muss nebenläufig programmiert werden.\ifootcite[31]{dekoster2016}
Eine Möglichkeit hierzu ist die Anwendung des Aktorenmodells -- ein konzeptionelles Entwurfsmuster für die Entwicklung von nebenläufigen Programmen.\ifootcite[235]{hewitt1973}

Diese Arbeit befasst sich mit den theoretischen Grundlagen des Aktorenmodells und der praktischen Implementierung eines Aktorensystems, im Rahmen der DHBW"=Veranstaltung \enquote{Anwendungsentwicklung 1}.
Zunächst folgt eine Einführung in die Theorie des Aktorenmodells.
Darauf werden die Anforderungen an die Anwendung genauer ausgeführt.
In \autoref{sec:implementation} wird die Implementierung des geforderten Aktorensystems beschrieben.
Abschließend werden in \autoref{sec:conclusion} die Inhalte dieser Arbeit nochmals zusammengefasst.
