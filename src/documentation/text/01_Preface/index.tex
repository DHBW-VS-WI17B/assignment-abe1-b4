Im Laufe der Zeit schreitet die Technologie im  Bereich der Prozessoren von Servern und Computern immer weiter voran.
Die Prozessoren werden zwar nicht schneller, haben jedoch immer mehr Kerne.
Um die verschiedenen Kerne der Hardware ausnutzen zu können, wird eine gleichzeitige Codeausführung benötigt.
Eine Möglichkeit hierzu ist das Aktorenmodell.

Das Aktorenmodell ist grundlegen ein konzeptionelles Modell, welches sich mit der gleichzeitigen Ausführung von Codes mit berücksichtigung von gewissen Regeln beschäftigt.\footnote{Vgl. https://www.brianstorti.com/the-actor-model/} 

Diese wissenschaftliche Ausarbeitung befasst sich mit den theoretischen Grundlagen des Aktorenmodells und der praktischen Entwicklung dessen im Rahmen der DHBW"=Veranstaltung \enquote{Anwendungsentwicklung 1}.
Zum Einstieg wird in \autoref{sec:theory} in die Theorie des Aktorenmodells eingeleitet.
In \autoref{sec:requirements} werden die Anforderungen an die Anwendung genauer ausgeführt, die in \autoref{sec:architecture} in Form der Architektur der Anwendung dargestellt werden.
Abschließend werden in \autoref{sec:conclusion} die Inhalte der wissenschaftlichen Arbeit nochmals zusammengefasst.
