In der Vergangenheit haben sich die Prozessoren für Server und Computer kontinuierlich weiterentwickelt, vorallem verfügen moderne Prozessoren über immer mehr Prozessorkerne.\ifootcite[1]{gruntz2010aktorenmodel}
Um die verschiedenen Kerne der Hardware ausnutzen zu können, muss nebenläufig programmiert werden.\ifootcite[31]{dekoster2016}
Eine Möglichkeit hierzu ist die Anwendung des Aktorenmodells -- ein konzeptionelles Entwurfsmuster für die Entwicklung von nebenläufigen Programmen.\ifootcite[235]{hewitt1973}

Diese Arbeit befasst sich, im Rahmen der DHBW"=Veranstaltung \enquote{Anwendungsentwicklung 1}, mit den theoretischen Grundlagen des Aktorenmodells und der praktischen Implementierung eines Aktorensystems.
Zunächst folgt eine Einführung in die Theorie des Aktorenmodells.
Darauf werden die Anforderungen an die Anwendung genauer ausgeführt.
In \autoref{sec:implementation} wird die Implementierung des geforderten Aktorensystems beschrieben.
Abschließend werden in \autoref{sec:conclusion} die Inhalte dieser Arbeit nochmals zusammengefasst.
