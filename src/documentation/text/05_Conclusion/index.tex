Ziel dieser Arbeit war der Transfer des Aktorenmodells in die Praxis, mit der Entwicklung eines Ticketshops für Veranstaltungen.
Zunächst wurde das Aktorenmodell vorgestellt, wobei definiert wurde, dass Aktoren ein Modell für die synchrone Berechnung paralleler, verteilter und mobiler Systeme darstellen und somit asynchron Nachrichten an andere Aktoren senden bzw.\ von anderen Aktoren empfangen können.
Bevor mit der Umsetzung gestartet werden konnte, wurden die einzelnen Anwendungsfälle modelliert und beschrieben.
Dadurch konnte gewährleistet werden, dass die gewünschte Qualität, die geforderten Funktionalitäten und Eigenschaften des Auftraggebers korrekt umgesetzt wurden.
Zu Begin der Implementierungsphase wurden als Basisarchitektur zwei Konsolenanwendungen definiert.
Darauf wurde ein Datenmodell erarbeitet, auf dessen Basis schließlich ein Aktorensystem entwickelt wurde, das sich aus Client"=, Event"=, und User"=Aktoren, sowie einem WriteToDb"=Aktor zusammensetzt.
Abschließend wurden alle Nachrichten"=Typen beschrieben, die zwischen den Aktoren versendet werden.
