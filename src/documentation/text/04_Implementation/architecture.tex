Grundsätzlich besteht die Implementierung aus zwei Konsolenanwendungen:


% client console app
% server console app
% write to db actor, damit keine gleichzeitigen schreibkonflikte entstehen
% immutable nachrichten, dass der state nicht ausversehen mutiert wird.

\begin{dhbwfigure}{%
    caption	= Aktorensystem,
    label	= fig:actor_system,
}
\begin{plantuml}
@startuml

package "Local" {
    [App] as app
    [ClientActor] as client
}

package "Remote" {
    [EventActor] as event
    [UserActor] as user
    [WriteToDbActor] as write
    database "Database" as db
}

app <-> client
client <-> event
client <-> user
event <-> write
user <-> write
user <-- db
event <-- db
write <--> db

@enduml
\end{plantuml}
\end{dhbwfigure}
