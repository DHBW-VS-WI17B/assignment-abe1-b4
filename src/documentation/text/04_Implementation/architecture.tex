Grundsätzlich besteht die Implementierung des geforderten Systems aus einer Client"= und einer Server"=Konsolenanwendung, die jeweils über Befehlszeilenargumente gesteuert werden können.
Mit dem Befehlszeilenargument \enquote{-h} können je Anwendung alle verfügbaren Befehle abgerufen werden.

\begin{dhbwfigure}{%
    caption	= Aktorensystem,
    label	= fig:actor_system,
}
\begin{plantuml}
@startuml

package "Local" {
    [App] as app
    [ClientActor] as client
}

package "Remote" {
    [EventActor] as event
    [UserActor] as user
    [WriteToDbActor] as write
    database "Database" as db
}

app <-> client
client <-> event : DotNetty TCP
client <-> user : DotNetty TCP
event <-> write
user <-> write
user <-- db
event <-- db
write <--> db

@enduml
\end{plantuml}
\end{dhbwfigure}

\autoref{fig:actor_system} zeigt den Aufbau des implementierten Aktorensystems.
Clientseitig instanziiert jeder Client einen eigenen \enquote{Client"=Aktor}, der die Schnittstelle zwischen der Konsolenanwendung und den restlichen Aktoren darstellt, wobei Akka.NET die Bibliothek \enquote{DotNetty} nutzt, um über TCP"=Verbindungen mit den entfernten Aktoren zu kommunizieren.
Serverseitig werden mehrere \enquote{User"=} und \enquote{Event"=Aktoren} instanziiert, die sich jeweils Anfragen bzgl.\ ihrer Domäne annehmen.
Sollte eine Anfrage schreibenden Zugriff auf die Datenbank benötigen, wird die Anfrage an einen nur einmal instanziierten \enquote{WriteToDb"=Aktor} weitergeleitet, der alle Anfragen gem.\ dem Aktorenmodell sequentiell abarbeitet, wodurch Dateninkonsistenzen verhindert werden können.
