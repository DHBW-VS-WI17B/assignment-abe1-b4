\autoref{fig:er_diagram} zeigt das Datenmodell der Implementierung als Entity"=Relationship"=Modell.
Es wurde bewusst drauf verzichtet aggregierbare Merkmale in das Datenmodell aufzunehmen.
So wird die Anzahl an verkauften Tickets für ein bestimmtes Event bspw.\ nicht als Merkmal eines Events gespeichert, sondern auf Anfrage anhand gespeicherter Tickets berechnet.

\begin{dhbwfigure}{%
    caption	= Entity-Relationship-Modell,
    label	= fig:er_diagram,
    float   = H
}
\begin{plantuml}
@startuml

skinparam linetype ortho

entity "Address" as address {
    * Id: Guid
    --
    * ZipCode: string
    * City: string
    * Street: string
    * HouseNumber: string
    --
    * UserId: Guid
}

entity "User" as user {
    * Id: Guid
    --
    * UserName: string
    * YearlyBudget: double
    --
    * AddressId: Guid
}

entity "Ticket" as ticket {
    * Id: Guid
    --
    * PurchaseDate: DateTime
    --
    * EventId: Guid
    * UserId: Guid
}

entity "Event" as event {
    * Id: Guid
    --
    * Name: string
    * Date: DateTime
    * Location: string
    * PricePerTicket: double
    * MaxTicketCount: int
    * MaxTicketsPerUser: int
    * SaleStartDate: DateTime
    * SaleEndDate: DateTime
}

user ||-|| address
user }o-|| ticket
event }o-|| ticket

@enduml
\end{plantuml}
\end{dhbwfigure}
