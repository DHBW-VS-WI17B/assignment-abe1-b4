Zur Realisierung werden von den in \autoref{sec:architecture} beschriebenen Aktoren Nachrichten ausgetauscht.
Damit der Zustand von Nachrichten nicht unabsichtlich verändert wird bei der Verarbeitung, sind alle Nachrichten als unveränderbare Objekte definiert.
Nachfolgend werden alle Nachrichten"=Typen kurz beschrieben:

\begin{itemize}[itemsep=-.5em,leftmargin=*]
    \item \textbf{CreateEventMessage}: Wird von der Client"=Applikation an den Client"=Aktor gesendet, um eine Veranstaltung zu erstellen.
    \item \textbf{GetAllEventsMessage}: Wird von der Client"=Applikation an den Client"=Aktor gesendet, um alle verfügbaren Veranstaltungen abzurufen.
    \item \textbf{GetEventByIdMessage}: Wird von der Client"=Applikation an den Client"=Aktor gesendet, um eine bestimmte Veranstaltungen abzurufen.
    \item \textbf{GetPurchasedTicketsMessage}: Wird von der Client"=Applikation an den Client"=Aktor gesendet, um alle bestellten Tickets abzurufen.
    \item \textbf{GetRemainingBudgetForCurrentYearMessage}: Wird von der Client"=Applikation an den Client"=Aktor gesendet, um das verbleibende Budget für das aktuelle Jahr abzurufen.
    \item \textbf{GetSoldTicketsMessage}: Wird von der Client"=Applikation an den Client"=Aktor gesendet, um die Anzahl an verkauften Tickets für eine Veranstaltung abzurufen.
    \item \textbf{InitStateMessage}: Wird von der Client"=Applikation an den Client"=Aktor gesendet, um einen Benutzer zu erstellen.
    \item \textbf{PersistStateMessage}: Wird innerhalb des Client"=Aktors gesendet, um den Identifikator des aktuellen Benutzers lokal zu persistieren.
    \item \textbf{PurchaseTicketsMessage}: Wird von der Client"=Applikation an den Client"=Aktor gesendet, um Tickets für eine Veranstaltung zu bestellen.
    \item \textbf{RestoreStateMessage}: Wird von der Client"=Applikation an den Client"=Aktor gesendet, um den lokalen Zustand nach einem Neustart der Client"=Applikation wiederherzustellen.
    \item \textbf{AddEventToDbRequest}: Wird von einem Event"=Aktor an den WriteToDb"=Aktor gesendet, um eine Veranstaltung in der Datenbank anzulegen.
    \item \textbf{AddEventToDbResponse}: Wird von WriteToDb"=Aktoren an den entsprechenden Event"=Aktor gesendet, um über das Ergebnis der angeforderten Schreiboperation zu informieren.
    \item \textbf{AddTicketsToDbRequest}: Wird von einem Event"=Aktor an den WriteToDb"=Aktor gesendet, um beliebig viele Tickets für eine Veranstaltung in der Datenbank anzulegen.
    \item \textbf{AddTicketsToDbResponse}: Wird von WriteToDb"=Aktoren an den entsprechenden Event"=Aktor gesendet, um über das Ergebnis der angeforderten Schreiboperation zu informieren.
    \item \textbf{AddUserToDbRequest}: Wird von einem User"=Aktor an den WriteToDb"=Aktor gesendet, um einen Nutzer in der Datenbank anzulegen.
    \item \textbf{AddUserToDbResponse}: Wird von WriteToDb"=Aktoren an den entsprechenden User"=Aktor gesendet, um über das Ergebnis der angeforderten Schreiboperation zu informieren.
    \item \textbf{CreateEventRequest}:  Wird von einem Client"=Aktor an einen Event"=Aktor gesendet, um das Anlegen einer neuen Veranstaltung anzufragen.
    \item \textbf{CreateEventSuccess}: Wird von einem Event"=Aktor an den entsprechenden Client"=Aktor gesendet, falls die angefragte Veranstaltung angelegt werden konnte.
    \item \textbf{CreateUserRequest}: Wird von einem Client"=Aktor an einen User"=Aktor gesendet, um das Anlegen eines neuen Nutzers anzufragen.
    \item \textbf{CreateUserSuccess}: Wird von einem User"=Aktor an den entsprechenden Client"=Aktor gesendet, falls der angefragte Nutzer angelegt werden konnte.
    \item \textbf{GetAllEventsRequest}: Wird von einem Client"=Aktor an einen Event"=Aktor gesendet, um alle gespeicherten Veranstaltungen abzurufen.
    \item \textbf{GetAllEventsSuccess}: Wird von einem Event"=Aktor an den entsprechenden Client"=Aktor gesendet, um alle gespeicherten Veranstaltungen zu übermitteln. 
    \item \textbf{GetEventByIdRequest}: Wird von einem Client"=Aktor an einen Event"=Aktor gesendet, um eine bestimmte Veranstaltung abzurufen.
    \item \textbf{GetEventByIdSuccess}: Wird von einem Event"=Aktor an den entsprechenden Client"=Aktor gesendet, um eine bestimmte Veranstaltung zu übermitteln.
    \item \textbf{GetPurchasedTicketsRequest}: Wird von einem Client"=Aktor an einen User"=Aktor gesendet, um alle bestellten Tickets anzufragen.
    \item \textbf{GetPurchasedTicketsSuccess}: Wird von einem User"=Aktor an den entsprechenden Client"=Aktor gesendet, um alle bestellten Tickets zu übermitteln.
    \item \textbf{GetRemainingBudgetForCurrentYearRequest}: Wird von einem Client"=Aktor an einen User"=Aktor gesendet, um das verbleibende Budget für das aktuelle Jahr anzufragen.
    \item \textbf{GetRemainingBudgetForCurrentYearSuccess}: Wird von einem User"=Aktor an den entsprechenden Client"=Aktor übermittelt, um das verbleibende Budget für das aktuelle Jahr zu übermitteln.
    \item \textbf{GetSoldTicketsRequest}: Wird von einem Client"=Aktor an einen Event"=Aktor gesendet, um die verkauften Tickets für die angefragte Veranstaltung anzufragen.
    \item \textbf{GetSoldTicketsSuccess}: Wird von einem Event"=Aktor an den entsprechenden User"=Aktor gesendet, um die verkauften Tickets für die angefragte Veranstaltung zu übermitteln.
    \item \textbf{PurchaseTicketsRequest}: Wird von einem Client"=Aktor an einen Event"=Aktor gesendet, um Tickets für eine Veranstaltung zu erwerben.
    \item \textbf{PurchaseTicketsSuccess}: Wird vin einem Event"=Aktor an den entsprechenden Client"=Aktor gesendet, um die erworbenen Tickets für die angefragte Veranstaltung zu übermitteln.
    \item \textbf{ErrorMessage}: Wird von einem Event"= oder User"=Aktor an einen Client"=Aktor gesendet, um auf einen Fehler bei der Verarbeitung einer Nachricht hinzuweisen.
\end{itemize}

In \autoref{app:actors_and_messages} werden die beschriebenen Nachrichten"=Typen mit dem Architekturdiagramm aus \autoref{fig:actor_system} zu einem Gesamtablaufdiagram ergänzt, das dem Verständnis der Implementierung weiter beitragen kann.
