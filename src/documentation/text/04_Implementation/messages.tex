Zur Realisierung werden von den in \autoref{sec:architecture} beschriebenen Aktoren Nachrichten ausgetauscht.
Damit der Zustand von Nachrichten nicht unabsichtlich verändert wird bei der Verarbeitung, sind alle Nachrichten als unveränderbare Objekte definiert.
Nachfolgend werden alle Nachrichten"=Typen kurz beschrieben:

\begin{itemize}[itemsep=-.5em,leftmargin=*]
    \item \textbf{CreateEventMessage}: Wird von der Client"=Applikation an den Client"=Aktor gesendet um eine Veranstaltung zu erstellen.
    \item \textbf{GetAllEventsMessage}: Wird von der Client"=Applikation an den Client"=Aktor gesendet um alle verfügbaren Veranstaltungen abzurufen.
    \item \textbf{GetEventByIdMessage}: Wird von der Client"=Applikation an den Client"=Aktor gesendet um eine bestimmte Veranstaltungen abzurufen.
    \item \textbf{GetPurchasedTicketsMessage}: Wird von der Client"=Applikation an den Client"=Aktor gesendet um alle bestellten Tickets abzurufen.
    \item \textbf{GetRemainingBudgetForCurrentYearMessage}: Wird von der Client"=Applikation an den Client"=Aktor gesendet um das verbleibende Budget für das aktuelle Jahr abzurufen.
    \item \textbf{GetSoldTicketsMessage}: Wird von der Client"=Applikation an den Client"=Aktor gesendet um die Anzahl an verkauften Tickets für eine Veranstaltung abzurufen.
    \item \textbf{InitStateMessage}: Wird von der Client"=Applikation an den Client"=Aktor gesendet um einen Benutzer zu erstellen.
    \item \textbf{PersistStateMessage}: Wird innerhalb des Client"=Aktors versandt um den Identifikator des aktuellen Benutzers lokal zu persistieren.
    \item \textbf{PurchaseTicketsMessage}: Wird von der Client"=Applikation an den Client"=Aktor gesendet um Tickets für eine Veranstaltung zu bestellen.
    \item \textbf{RestoreStateMessage}: Wird von der Client"=Applikation an den Client"=Aktor gesendet um den lokalen Zustand nach einem Neustart der Applikation wiederherzustellen.
    \item \textbf{AddEventToDbRequest}: Wird von dem Event"=Aktor an den Datenbank"=
    \item \textbf{AddEventToDbResponse}:
    \item \textbf{AddTicketsToDbRequest}:
    \item \textbf{AddTicketsToDbResponse}:
    \item \textbf{AddUserToDbRequest}:
    \item \textbf{AddUserToDbResponse}:
    \item \textbf{CreateEventRequest}:
    \item \textbf{CreateEventSuccess}:
    \item \textbf{CreateUserRequest}:
    \item \textbf{CreateUserSuccess}:
    \item \textbf{GetAllEventsRequest}:
    \item \textbf{GetAllEventsSuccess}:
    \item \textbf{GetEventByIdRequest}:
    \item \textbf{GetEventByIdSuccess}:
    \item \textbf{GetPurchasedTicketsRequest}:
    \item \textbf{GetPurchasedTicketsSuccess}:
    \item \textbf{GetRemainingBudgetForCurrentYearRequest}:
    \item \textbf{GetRemainingBudgetForCurrentYearSuccess}:
    \item \textbf{GetSoldTicketsRequest}:
    \item \textbf{GetSoldTicketsSuccess}:
    \item \textbf{PurchaseTicketsRequest}:
    \item \textbf{PurchaseTicketsSuccess}:
    \item \textbf{ErrorMessage}:
\end{itemize}

In \autoref{app:actors_and_messages} werden die beschriebenen Nachrichten"=Typen mit dem Architekturdiagramm aus \autoref{fig:actor_system} zu einem Gesamtablaufdiagram ergänzt, das dem Verständnis der Implementierung weiter beitragen kann.
