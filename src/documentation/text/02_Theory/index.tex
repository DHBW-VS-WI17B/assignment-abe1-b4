Aktoren stellen ein Modell für die gleichzeitige Berechnung paralleler, verteilter und mobiler Systeme dar.
Ein Aktor ist ein autonomes Objekt, das asynchron arbeitet und asynchrones Empfangen und Senden von Nachrichten an andere Aktoren ermöglicht.
Zudem können Aktoren weitere Aktoren erzeugen und einen eigenen Zustand verwalten.\ifootcite[1]{karmani2011}
Jeder Aktor verfügt dabei über einen Posteingang, eine Adresse und ein Verhaltensmuster, das vom aktuellen Zustand des Aktors abhängt.
Beim Empfang einer Nachricht können Aktoren auf drei Arten reagieren:
Nachrichten an sich selbst oder andere Aktoren verschicken, neue Aktoren erzeugen oder den eigenen Zustand ändern.
Der Nachrichtenaustausch erfolgt asynchron, was bedeutet, dass der Sender nach dem Versenden einer Nachricht sofort mit einer anderen Aktion fortfahren kann, also nicht blockiert.\ifootcite{doc.akka.io_actorsintro}
Da ein Aktor eingehende Nachrichten sequentiell abarbeitet, entstehen keine Inkonsistenzen, wie es z.B.\ beim parallelen Zugriff durch mehrere Prozesse theoretisch möglich ist.

\begin{dhbwfigure}{%
    caption	= Aktorenmodell Funktionsprinzip,
    label	= fig:actor_model_functionality,
    float   = H
}
\begin{plantuml}
@startuml

scale 0.65

actor "Aktor 1" as a1
actor "Aktor 2" as a2
actor "Aktor 3" as a3

a1 -> a2: Nachricht A
a3 -> a2: Nachricht B kommt in die Warteschlange
a2 -> a2: Nachricht A wird verarbeitet
a2 -> a2: Nachricht B wird der Warteschlange entnommen
a2 -> a2: Nachricht B wird verarbeitet

@enduml
\end{plantuml}
\end{dhbwfigure}

\autoref{fig:actor_model_functionality} zeigt, wie Nachrichten zweier Aktoren nach dem \enquote{First-In-First-Out}"=Prinzip sequentiell verarbeitet werden und so der Zustand von Aktor 2 konsistent bleibt.
